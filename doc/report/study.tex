\section{Understanding Performance Bugs} 
\label{s:study}
%To the best of our knowledge, there is currently little knowledge about 
%
In this section, we present an empirical study on several known performance bugs in mainstream network operating systems 
to help understand their characteristics. % of the performance bugs. % in Markdown compilers.
%

\subsection{Data Collection}
\label{s:study-bug}

\iffalse
\begin{table}[t]
\centering
    \caption{The existing performance bugs included in our study.
    %Lang. means the underlying programming language for implementing the Markdown compiler.
    }
\label{table:study-dataset}
\small
%\resizebox{0.47\textwidth}{!}{
\begin{tabular}{lcc}
   \toprule
      Software  & \# Bugs & Time Periods \\
    \midrule
    Cicso IOS \cite{cicso-ios} & 58 & 10/26/2001 - 01/13/2021 \\
     & & 01/14/2017 - 12/20/2020 \\
    MD4C & 6 & 03/10/2019 - 09/10/2019\\
    commonmark.js & 9 & 09/28/2017 - 08/13/2019 \\
    markdown-it & 8 & 08/14/2019 - 11/20/2020\\
%    \midrule
%    Total &  & (49) 52 & 10/26/2014 - 12/20/2020\\
   \bottomrule
\end{tabular}
%}
\end{table}
\fi

%As shown in \autoref{table:study-dataset}, 
%
We investigate performance bugs in the Cicso IOS \cite{cicso-ios}, 
%
which is one of the most popular network operating systems.
%
Ideally, if more network operating systems are included in the study we can present a more comprehensive characteriazation.
%
However, due the the limited time and human labours allowed for this work,
%
we have to somehow restrict the study scope.
%
Nevertheless, this work, as an early stage studying performance bugs in network operating systems, shall provide some general inspirations for future network system security analysis.

We manually collected 58 distinct performance bugs from the database of Common Vulenrabilities and Exposures (CVE) \cite{cve}.
%
The detailed information about the vulnerabilities included in our study is presented in \autoref{cve-details}.
%
Some bug reports might be duplicates.
%
We remove a duplicate bug from our dataset if it has a similar cause as another one,
%
though it can have a different exploit vector.
%
In summary, we obtained 49 distinct performance bugs by removing \XXX duplicates from the ones reported from November 2001 to January 2021.
%
We found all these performance bugs were abusing the CPU resources or memory resources.
%
%This suggests CPU resource exhaustion performance bugs are the dominant type of performance bugs.

We next characterize the performance bugs and present our findings.

\subsection{Bug Disclosure Over Time}
\label{s:study-time}

\begin{figure}[t]
    \centering
    \includegraphics[width=0.45\textwidth, trim =5 0 0 0,clip]{fig/disclosure-time.pdf}
    \caption{The number of performance bug reports over time.
    }
    \label{fig:disclosure-time}
\end{figure}

To understand the trend of performance bugs, we analyze the disclosure time of them.  
We depict the number of performance bugs along the time they were disclosed in \autoref{fig:disclosure-time}.
%
We observe that few bugs were reported before early 2004, and the number of reported bugs had been gradually growing from early 2018 till late 2020.
%
In particular, 28 (57.14\%) out of the 49 performance bugs were disclosed from April 2018 till December 2020 (22 months);
%
21 (42.86\%) bugs were reported from August 2014 to December 2017 (41 months).
%
It reveals that such bugs had been gradually drawing the attention from the compiler developers and security analysts.
%

\subsection{Root Causes}
\label{s:study-root-causes}
%
Identifying the common root causes of real-world performance bugs can benefit potential future research and software developments.
%
We manually analyzed the performance bugs and successfully figured out the root causes for \XX bugs.
%
We classify the root causes into three categories.
%
A bug is assigned to multiple categories if it has multiple major causes.


\PN{R1:} \emph{Super-linear algorithms.}
Some normal algorithms implemented in Markdown compilers have super-linear worst-case complexity \cite{slowfuzz, perffuzz}.
%
Attackers can craft inputs to trigger the worst-case behaviors and lead to performance issues.
%
%For instance, some compilers use regular expressions to match inputs, which are vulnerable to ReDoS attacks \cite{redos} that trigger excessive numbers of backtracking in the matching process.
%
The majority (25 out of 39) of bugs were related to such worst-case behaviors.
%\WM{Check this number again after moving ReDoS to R2.}
%
%Specifically, the abnormal backtracking logic in Markdown compilers can be further divided into two groups:
%
%
%Known performance bugs in this category exploit either the design of Markdown compilers or the vulnerable regular expression and can lead to polynomial or exponential complexity compilation time.
%
%What is worse, the intended normal functionality of the Markdown compilers can be abused by crafted inputs for triggering worst-case behaviors, \eg{,} excessive number of backtracking.
%
%This includes triggering cataphoric backtracking in regular expression matches and the similar logic design inside the Markdown compilers can be abused with specially crafted inputs.
%
%
%If the input size is not limited, the Markdown compilers can easily hang for up to several hours.
%
%
%27 out of 39 performance bugs belong to this category.
%Specially crafted inputs abuse the backtracking behaviors and lead.
%
%We investigate the specially crafted inputs that lead to such worst-case behaviors. %and present two bugs in this category.
%We discuss two primary kinds of algorithms that exhibit the super-linear worst-case complexity we find. % in the Markdown compilers.
%
%They can be roughly divided into two groups:
%
%(1) long open tokens without close tokens,
%
%and (2) long early open tokens and late close tokens.
%

Some Markdown syntaxes (\eg{,} links, emphasis and strong emphasis, HTML blocks) are related to the language's context-sensitive features.
%
%We notice that the performance bugs are quite related to the logic (syntax) of the Markdown language,
%
%in particular, its context-sensitive features.
%
As discussed in \autoref{s:background-compiler},
supporting context-sensitive features in Markdown requires the compilers to backtrack, which could take more than linear time.
%
The backtracking strategies can easily be abused with crafted inputs hence lead to performance issues.
%
%\WM{Check the following. The total number 25+14 are larger than the above 27.}
For instance, links were the primary vulnerable syntax in Markdown compilers,
%
where 11 of the known performance bugs could be exploited with special inputs with links.
%
%Usually, they can happen together with other syntax components.
%
%
%Emphasis and strong emphasis is the second vulnerable Markdown syntax.
%HTML blocks are also the most vulnerable Markdown syntax component in Markdown compilers.
%
Similarly, 8 of the bugs were caused by the buggy emphasis and strong emphasis handlers.
%
Our study reveals that the implementation of the context-sensitive features in the Markdown compilers are prone to containing performance bugs.
%For example, the backtracking from wrong options can be abused and result in performance issues.



One typical input pattern that exploits the context-sensitive syntax handler to trigger performance bugs is \emph{many open tokens}.
%
This pattern can lead the compilers to repeatedly search a close token towards the end of the input string for each such open token, and also force the compilers to backtrack to correct wrong options the compilers have selected.
%
%Regarding the number of repetitions, $n$, the time complexity can be $O(n^2)$ or higher.
%Some logic design of Markdown compilers produces similar backtracking behaviors when a temporal mismatch is detected.
%
For example, deeply nested CDATA block open delimiters can result in an excessive compilation time.
%
%When Markdown compilers detect the first CDATA block open delimiter (\ie{,} \blstinline{<![CDATA[}), they have to search to the end until a matched CDATA block close delimiter (\ie{,} \blstinline{]]>}) is found.
%
%When the match cannot be found, the Markdown compilers perform backtracking to the current CDATA open delimiter.
%
When fed with $n$-nested CDATA block open delimiters (\eg{,} \blstinline{'<![!CDATA[<![CDATA[<![CDATA[...'}) that are not closed with the corresponding close delimiters (\ie{,} \blstinline{']]>'}) or are closed in the end of the input string,
%
%can backtrack repeatedly on every open delimiter and in total for $n$ times.
the compilers need to compare with all tokens in the input string to determine if an open delimiter can be closed or not.
%
Once the compilers find an open delimiter cannot be closed, they switch to other possible options for that delimiter next, for instance, the open delimiter \blstinline{'<!'} in \blstinline{'<!A>'}, which cannot be closed either.
%
%Each time, the compilers have to search till the end of the input string.
%
Thus the time for handling such input strings is at least in polynomial time complexity.
%
By providing a long input with many such open tokens, it is simple to cost the compiler several-second or even more execution time.
%Depending on how such backtracking strategies are designed, such inputs can cause polynomial or higher complexity performance issues, easily resulting in seconds of execution time.


\PN{R2:} \emph{Inefficient code.}
%
Some inefficient code in the Markdown compilers could also lead to performance issues.
%
For instance, some functions do not coordinate well for certain functionalities.
%
We find that 9 performance bugs were caused by such inefficient code.
%
Unlike the algorithms in R1, such performance issues could be addressed by optimizing the inefficient code.
%
However, each problem needs to be separately analyzed and fixed, which could be time-consuming.
%
We next discuss an example of such inefficient code.
%

Minor performance issues in individual problematic functions could accumulate when the given inputs can repeatedly trigger the execution of such functions.
%
For example, in one bug, cmark calls \blstinline{S_find_first_nonspace()} to find the first non-space character from the current offset in a line. %, which has a minor performance issue by
%
%The function went back from the current offset backwardly to find the first non-space character,
The function in a second call would still search from the initial position,
even if in a previous call it has already recognized the location of the first non-space character.
%
This means some function calls to \blstinline{S_find_first_nonspace()} sometimes were unnecessary.
%
Crafted inputs with lots of complicated and nested indents could result in repeated invocations of this function and cause performance bugs.
%
The problem, however, can be solved by using better strategies like cashing the positions of the previously found non-space characters.
%

\iffalse
Second, some Markdown compilers match input strings using regular expressions (regex).
%
Some vulnerable regex could lead to performance bugs if the regex engine needs to backtrack (in exponential time) when matching some specific inputs.
%
A failed matching attempt can lead the engine to backtrack with multiple choices.
%
Thus the total number of possible backtracking paths is exponential if the match cannot be found eventually for every input symbol.
%
This is also known as regular expression denial-of-service (ReDoS) \cite{freezing, shen2018rescue, redosimpact, wustholz2017static}.
%
For instance, one bug in commonmark.js was caused by using a regex containing a vulnerable subexpression \blstinline{(.|\\\\)*]}\footnote{The vulnerable subexpression is simplified for demonstration.}
to match link labels (\eg{}, the \blstinline{[demo]} in \blstinline{'[demo](url 'title')'}).
%
Its syntax is to match any non empty character \blstinline{.} or a single blackslash \blstinline{\\\\} for any number of times \blstinline{*}, followed by a closing bracket \blstinline{]}.
%
%
Suppose the compiler uses this regex to match an input string \blstinline{'\\\\\\\\...'} that contains $n$ backslash characters.
%
The matching would eventually fail because the input string does not contain any closing bracket.
%
But before the failure, the regex engine would check all possible matches for the previous part \blstinline{(.|\\\\)*}.
%
%
Each backslash character in the input string can match either \blstinline{.} or \blstinline{\\\\}, so $n$ characters would cause the engine to backtrack with $2^n$ possible paths.
%
Nevertheless, such performance bugs could be addressed by fixing the vulnerable regex.
\fi
%

\PN{R3:} \emph{Implementation-specific issues.}
Other causes of the bugs are specific to the compiler implementations or designs.
%
Some compilers overlooked part of the CommonMark specification, for example, Unicode support.
%
This can lead to infinite loops when unexpected inputs are provided to the compilers.
%
Some other bugs in this category were caused by wrong data structures.
%
5 performance bugs fall into this category.

\begin{comment}
\subsection{Markdown Syntax and Performance Bugs}
\label{s:study-markdown}

\begin{table}[t]
\caption{Top 4 types of vulnerable Markdown syntax}
\label{tab:bug-syntax}
\centering
\small
\resizebox{0.47\textwidth}{!}{
\begin{tabular}{lcll}
    \toprule
    Markdown syntax & Bugs & Examples  & Root causes\\
    \midrule
    Links & 25 & \blstinline{'[ (]( [ (]( ...'} & \textbf{R1} \textbf{R2} \textbf{R3} \\
    (Strong) Emphasis &14 & \blstinline{'a\_ a\_ ...'} & \textbf{R1} \textbf{R3}\\
    HTML blocks & 9 & \blstinline{'<!\[CDATA\[ <![CDATA[...'} & \textbf{R1} \\
    Fenced code blocks & 5 & \blstinline{'```\\ncode\\n...'} & \textbf{R1}\\
   \bottomrule
\end{tabular}
}
\end{table}

%

We investigate the relationship between the Markdown syntax and the compiler performance bugs.
%
We present the top 4 types of vulnerable Markdown syntaxes of the known performance bugs in \autoref{tab:bug-syntax}.
%
A bug is classified into one or multiple syntax groups if it is caused by the inefficient or incorrect implementations of the syntax(es).
%
%A bug can be classified into several syntax groups if it breaks several syntax handlers.
%
We also list an example for each syntax group in the third column of \autoref{tab:bug-syntax}.
%


We also show the main root causes in each syntax component in the last column of \autoref{tab:bug-syntax}.
%
We include the root cause for a syntax group if at least one bug in the group has that root cause.
%
The results show that unlimited backtracking behaviors were quite common---all the top 4 buggy Markdown syntax components contained bugs with such a root cause.
\WM{<- Check this backtracking.}

\end{comment}

\iffalse
\subsection{Patches of Performance Bugs}
\label{s:study-patch}
We investigate the patches of performance bugs in Markdown compilers to understand how they were addressed.
%
We manage to identify the bug fix patterns for 28 performance bugs.
%
We present our findings below.
%

\PN{P1:} \emph{Enforcing limits.}
%
The most common patch pattern is to add limits for certain conditions such as the maximum depth of the nested structure,
%
although the CommonMark specification does not explicitly specify any such limits.
%
%
When such limits are reached, the compilers directly regard the rest unanalyzed inputs as plain text.
%
%Most uses of Markdown compilers normally do not exceed such limits,
%
%\eg{}, the depth of nested parenthesis seldom reaches thousands of layers.
%
Enforcing limits can prevent excessive CPU usages caused by the worst-case exploitation of too large test cases.
%By adding a limit,
%
However, 
the intended functionality might be violated.
%
It is also difficult to set a correct limit to prevent all attacks while not breaking some unusual yet legitimate inputs.
%the Markdown compilers will abort the tasks and process other tasks if a certain limit is reached.
%
%Such a strategy is simple yet useful.
%
Such a strategy has been applied to patch 13 out of the 28 bugs we investigate.

\PN{P2:} \emph{Logic changes.}
Logic changes sometimes are necessary as the bugs are caused by the incorrect coordination among multiple program components and functions.
%
Some inefficient code snippets need to be further optimized to eliminate the underlying performance issues.
%
For some other performance bugs caused by incorrect regular expressions,
compiler developers mainly review and rewrite the regular expressions.
\fi
