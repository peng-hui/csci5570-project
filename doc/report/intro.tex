\section{Introduction}
\label{s:intro}
Network Operating Systems (NOSs) connect multiple computers and devices and manage network resources in the settings of Software Defined Networking (SDN).
%
NOSs manage multiple requests concurrently and provide the security in multiuser environments. 
%
To date, there are many NOSs being proposed and matained by network service vendors.
%
For example, Cisco Internetwork Operating System is a family of network operating systems used on many Cisco Systems routers and current Cisco network switches \cite{edgeworth2014ip};
DD-WRT \cite{dd-wrt} is a Linux based open-sourced NOS suitable for various WLAN routers and embedded systems.
%
%

Due to their popularity and critical uses, the bugs in these network operating systems can lead to severe security consequences.
%
In particular, network operating systems could have performance bugs, which cause excessive resource consumption and could negatively affect user experiences.
%
They can be further leveraged by attackers for launching application-layer denial-of-service (DoS) attacks \cite{crosby2003algodos}.
%
By specially crafting inputs to trigger a performance bug in the network operating systems,
%
the attacker can exhaust the server's resources (\eg{,} memory and CPU) and make the services unavailable to normal users. 

%
To the best of our knowledge, there currently has little knowledge about the prevalence of performance bugs in the wild.
%performance bugs.
%
Prior studies on compilers generally focus on memory corruptions,
%
whereas performance bugs, especially in close-sourced network operating systems, are not well investigated and understood yet.
%
Some works studied performance issues in the regular expression engines \cite{shen2018rescue,wustholz2017static}, desktop software \cite{perfbugstudy}, and Android applications \cite{liu2014characterizing}.
%
Network operating systems, however, have not been covered.
%
To this end, we conduct a comprehensive study of the performance bugs in network operating systems to answer the following research questions:
%
\begin{itemize}[itemsep=0.5ex, parsep=0ex, leftmargin=3mm]
    \item What are the main categories of the bugs?
    \item What are the main factors that cause bugs?
    \item How widespread are the bugs in network operating systems?
    \item How severe are the bugs in network operating systems?
\end{itemize}

We empirically analyze 51 known performance bugs in mainstream network operating systems and thoroughly summarize their characteristics.
%
We observe that there has been a continuous growth in the number of reported performance bugs in the past 3 years.
%
We further identify that the ways that network operating systems handle the language's context-sensitive features are the dominant root cause of the performance bugs.
%
We reveal that the developers of network operating systems usually mitigate such exploitation by enforcing a hard limit for the maximum number of backtracking for certain tasks, which, however, limits the intended functionality of the network operating systems.


In this work, we also explore detecting performance bugs in network operating systems.
%
To the best of our knowledge, there is not any tailored tools specially designed for detecting performance bugs in network operating systems.
%
Fuzzing \cite{slowfuzz, hotfuzz, perffuzz} eliminates the limitations of static analysis techniques 
(\eg{,} high false positives) 
\cite{perfbugstudy, nistor2013toddler,nistor2015caramel}.
%
It has become the go-to approach with thousands of vulnerabilities in real-world software.
%
Without a doubt, performance bugs in network operating systems can be fuzzed as well.
%
%However, existing fuzzers \cite{slowfuzz, perffuzz} for performance bugs are not aware of special \emph{contex} of network operating systems and cannot effectively generate inputs to thoroughly exercise them.
%
%Therefore, besides the bit/byte level mutation \cite{afl}, we summarize the rules of the inputs. 
%
%We develop a syntax-tree based mutation strategy \cite{superion} to preserve and extend useful rules during the input mutation.
%
%This helps to generate inputs more effectively.
%
%

We mainly focus on CPU exhaustion performance bugs.
%
To detect them, we monitor the program execution under the generated inputs.
%
We employ a statistical model using Chebyshev inequality to label abnormal cases as performance bugs.
%
This can potentially cause duplicate bug reports as multiple bug reports with only slight difference in the inputs actually trigger the same bug.
%
It is time-consuming and impractical to leverage human efforts to manually de-duplicate them.
%
Existing bug de-duplicating methods using coverage profile and call stacks are inaccurate and not applicable to performance bugs.
%
For example, it is hard to obtain an accurate and deterministic call stack that reveals the situation when a performance bug is triggered.
%
Therefore, we propose a execution trace similarity comparison algorithm to de-duplicate the reports.
%
Specifically, we represent the execution trace of each report into a vector;
%
we compute the cosine similarity between vector pairs and classify vectors (bug reports) with high similarity as the same bug.
%

We integrate abovementioned techniques into \sys.
%
We demonstrated \sys outperformed the state-of-the-art works \cite{slowfuzz, perffuzz} \emph{better} performance slowdown, and \emph{higher} code coverage.
\sys is highly effective in generating test cases to trigger new code in the testing network operating systems.
%
Though we have not identified any new bugs yet after 6-hour testing, we plan to further keep \sys running for more time.
%
Hopenfully, our techniques can help find new bugs in the future.

In summary, we make the following contributions in this work.
\begin{itemize}
    \item
        We conduct a systematic study on performance bugs in network operating systems.
        %
        We present an empirical understanding of the performance bugs and the patches.
        %
    \item We develop a new system, \sys, to effectively generate useful inputs to detect performance bugs in network operating systems.
        %
    \item We demonstrate \sys can significantly outperform the state-of-the-art works.
\end{itemize}
