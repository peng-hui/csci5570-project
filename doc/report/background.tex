\section{Background}
\label{s:background}

\subsection{Performance Bugs}
\label{s:background-bug}
%\TODO{Have not had time to rephrase the decriptions from 'parser'->'compiler'}
Performance bugs in a program could degrade its performance and waste computational resources.
%
Usually, people define performance bugs as software defects where relatively simple \emph{source-code} changes can significantly optimize the execution of the software while preserving the functionality \cite{perfbugstudy, killian2010finding, s2e}.
%
There can be several different performance issues regarding different categories of resources.
%
For example, some performance bugs could cause excessive CPU resource utilization, resulting in unexpectedly longer execution time; 
%
some other bugs could lead to huge memory consumption because of uncontrolled memory allocation and memory leak \cite{wen2020memlock}.
%

%they 
Performance bugs lead to reduced throughput, increased latency, and wasted resources in software.
%
They particularly impact the end-user experiences.
%
What is worse, when a buggy application is deployed on the web servers, the bugs can be exploited by attackers for denial-of-service attacks, which can impair the availability of the services \cite{rampart}.
%
In the past, performance bugs have caused several publicized failures, causing many software projects to be abandoned \cite{perfbugstudy, lessons}.

\subsection{Network Operating Systems}
\label{s:background-nos}
Network operating system is a computer operating system that facilitates to connect and communicate various autonomous computers over a network. 
%
An autonomous computer is an independent computer that has its own local memory, hardware, \etc{}
%
It is self capable to perform operations and processing for a single user. 
%
Network operating systems can be embedded in a router or hardware firewall that operates the functions in the network layer \cite{al2001dialoguer}.
%
Typical real-world network operating systems include Cisco IOS \cite{cisco-ios}, DD_WRT \cite{dd-wrt}, Cumulus Linux \cite{cumulus-linux}, \etc{}


