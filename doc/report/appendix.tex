\begin{appendices}
    \pagebreak
\section{CVEs in Our Study}
\label{cve-details}
    We present the CVEs included in our study (\autoref{s:study}) in \autoref{tab:cve-details}.
\begin{table*}[]
    \caption{The CVEs and the empirical study. More details about each CVE can be found via searching the CVE ID.}
\label{tab:cve-details}
\centering
\small
%\resizebox{0.47\textwidth}{!}{
\begin{tabular}{lcclcc}
    \toprule
    CVE ID & Disclosure Time & Root Causes &
    CVE ID & Disclosure Time & Root Causes \\
    \midrule
    CVE-2021-1267 & 01/13/2021 & R1 & 
    CVE-2020-3567 & 10/07/2020 & R2 \\
    CVE-2020-3175 & 02/26/2020 & R2 &
    CVE-2020-3164 & 03/04/2020 & R2 \\
    CVE-2019-1967 & 08/28/2019 & R2 &
    CVE-2019-1957 & 08/07/2019 & R1\\
    CVE-2019-1947 & 02/19/2020 & R2 &
    CVE-2019-1806 & 05/15/2019 & R1\\
    CVE-2019-1721 & 04/17/2019 & R1 &
    CVE-2019-1720 & 04/17/2019 & R1\\
    CVE-2019-1718 & 04/17/2019 & N/A &
    CVE-2019-12698 & 10/02/2019 & R1 R2 \\
    CVE-2018-15462 & 05/01/2019 & R1 &
    CVE-2018-15460 & 01/09/2018 & N/A \\
    CVE-2018-15454 & 10/31/2018 & R1 & 
    CVE-2018-15388 & 05/01/2019 & R2\\
    CVE-2018-0272 & 04/18/2018 & R1 &
    CVE-2018-0257 & 04/18/2018 & R2 \\
    CVE-2018-0230 & 04/18/2018 & R2 &
    CVE-2018-0228 & 04/18/2018 & R2 \\ % 20
    CVE-2018-0177 & 03/28/2018 & R1 &
    CVE-2018-0094 & 01/17/2018 & R1 \\
    CVE-2017-6641 & 05/17/2017 & R1 R2  &
    CVE-2017-3885 & 04/05/2017 & R1\\
    CVE-2017-3820 & 02/01/2017 & R1 &
    CVE-2017-12244 & 10/04/2017 & R1 \\
    CVE-2017-12237 & 08/27/2017 & R2 & 
    CVE-2017-12211 & 09/06/2017 & R1 \\
    CVE-2016-6301 & 08/03/2016 & R2 & 
    CVE-2016-1483 & 09/14/2016 & R1\\ % 30
    CVE-2016-1440 & 06/27/2016 & R2 & 
    CVE-2016-1343 & 04/28/2016 & R2 \\
    CVE-2015-6386 & 09/28/2015 & R2 & 
    CVE-2015-6295 & 09/15/2015 & N/A \\
    CVE-2015-4283 & 07/20/2015 & R1 & 
    CVE-2015-0772 & 06/09/2015 & N/A \\
    CVE-2015-0765 & 06/03/2015 & N/A &
    CVE-2015-0754 & 05/27/2015 & R2 \\
    CVE-2015-0744 & 05/29/2015 & N/A & 
    CVE-2015-0617 & 02/16/2015 & R2 \\ % 40
    CVE-2015-0581 & 01/28/2015 & N/A & 
    CVE-2014-3353 & 09/02/2014 & N/A \\
    CVE-2014-3293 & 10/27/2014 & R2 & 
    CVE-2013-5566 & 11/06/2013 & R1 \\
    CVE-2013-3453 & 08/21/2013 & R1 &
    CVE-2013-1230 & 04/30/2013 & R1 R2 \\
    CVE-2012-3079 & 05/30/2012 & N/A &
    CVE-2012-2472 & 05/07/2012 & N/A \\
    CVE-2011-3287 & 08/29/2011 & R1 & 
    CVE-2010-4670 & 01/06/2011 & N/A \\
    CVE-2007-3698 & 07/25/2007 & N/A \\
    \bottomrule
\end{tabular}
%}
\end{table*}

\section{Artifact Instructions}
\subsection{Download and Install \sys}
We make \sys publically for the assessment of this project at \url{https://github.com/peng-hui/csci5570-project}.
%
Interested readers feel free to download, install, and verify our results.
%
Below are some details.

\begin{enumerate}
    \item Download or clone the project to your machine or server.
        The code is tested on Debian GNU/Linux 9.
    \item Compile the AFL via
        \cc{cd csci5570-project/afl \&\& make}.
    \item Compile the instrumented compile via
        \cc{cd llvm_mode \&\& make}.
\end{enumerate}

\subsection{Instrument the Testing Software}
This part varies depending on the testing software.
%
Please repalce the CC compiler with \cc{csci5570-project/afl/afl-clang-fast} and CXX compiler with \cc{csci5570-project/afl/afl-clang-fast++}.
%
Make sure you can compile the testing software successfully.

\subsection{Run \sys}
Suppose you have compiled \cc{cmark}.
Then you only need to run \cc{/afl-fuzz -p -i seeds -o out/ -N 64 ./cmark @@}
The \cc{seeds} is optional.
%
The \cc{out} is the result directory.

\subsection{Report De-Duplication}
Simply run \cc{python3 cosine-similarity.py}.
You might have to adjust the result directory accordingly.
\end{appendices}
